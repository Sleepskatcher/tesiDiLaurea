% !TEX encoding = UTF-8
% !TEX TS-program = pdflatex
% !TEX root = ../tesi.tex

%**************************************************************
\chapter{Conclusioni}
\label{cap:conclusioni}

%**************************************************************
\section{Raggiungimento degli obiettivi}
\label{sec:raggiungimento-obiettivi}

Per quanto riguarda gli obiettivi prefissati ad inizio stage si può evincere dalla tabella sottostante che sono stati soddisfatti tutti gli obiettivi ad esclusione dell'obiettivo facoltativo legato al JWT token. Questa mancanza è dovuta al fatto che in fase di progettazione è stato deciso dall'azienda di gestire unicamente tramite il back end le transazioni e i wallet address e questo mi ha impossibilitato ad implementare la gestione del JWT Token.

\begin{table}[H]
	\caption{Tabella dello stato di soddisfacimento degli obiettivi}
	\label{tab:obiettivi-raggiunti}
	\renewcommand{\arraystretch}{1.6}
	\begin{center}
	\begin{tabularx}{0.4\textwidth}{c|c}
		\hline\hline
		\textbf{Obiettivo} & \textbf{Descrizione}\\
		\hline
		O01 & Soddisfatto\\
		\hline
		O02 & Soddisfatto\\
		\hline
		O02 & Soddisfatto\\
		\hline
		D01 & Soddisfatto\\
		\hline
		F01 & Non soddisfatto\\
		\hline
	\end{tabularx}
	\end{center}
\end{table}%

Data l'impossibilità di soddisfare il requisito facoltativo a causa di una scelta progettuale, in accordo con l'azienda, ho prodotto le seguenti maschere non previste dal piano di lavoro:
\begin{itemize}
	\item homepage;
	\item registrazione;
	\item modifica dei dati personali;
	\item modifica della password;
	\item pagina dell'utente.
\end{itemize}

Inoltre ho iniziato la fase di testing anch'essa non compresa nel piano di lavoro.

%**************************************************************
\section{Conoscenze acquisite}
\label{sec:conoscenze-acquisite}

Durante questo periodo di stage ritengo di aver imparato molto. Sono infatti riuscita a consolidare e migliorare le mie conoscenze riguardanti il linguaggio javascript. come dico che ho imparato ad usare vue in maniera non marginale, ma sono entrata nel dettaglio utilizzando due plugin molto importanti ed utilizzati.\\
Ho potuto inoltre migliorare la mia capacità di affrontare e risolvere un problema senza fare affidamento totale a chi è più esperto di me, sono diventata più indipendente come lo dico?\\
Infine, avendo dovuto lavorare a stretto contatto con i colleghi stagisti, ho migliorato ulteriormente la mia capacità di lavorare in gruppo e di comunicazione. Infatti ho capito che queste due caratteristiche sono elementi molto importanti per poter riuscire a sviluppare al meglio un prodotto software.

%**************************************************************
\section{Valutazione personale}
\label{sec:valutazione-personale}

Alla fine di questa esperienza posso ritenermi soddisfatta del percorso svolto.
