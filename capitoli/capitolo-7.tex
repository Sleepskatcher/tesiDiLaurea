% !TEX encoding = UTF-8
% !TEX TS-program = pdflatex
% !TEX root = ../tesi.tex

%**************************************************************
\chapter{Conclusioni}
\label{cap:conclusioni}

\textit{In questo capitolo verranno descritto un resoconto conclusivo del progetto di stage con una valutazione degli obiettivi raggiunti e del percorso di stage effettuato.}

%**************************************************************
\section{Raggiungimento degli obiettivi}
\label{sec:raggiungimento-obiettivi}

Per quanto riguarda gli obiettivi prefissati ad inizio stage si può evincere dalla tabella sottostante che sono stati soddisfatti tutti gli obiettivi ad esclusione dell'obiettivo facoltativo legato al \textit{JWT token}. Questa mancanza è dovuta al fatto che in fase di progettazione è stato deciso dall'azienda di gestire unicamente tramite il \gls{back endg} le transazioni e i \textit{wallet address} e questo mi ha impossibilitato ad implementare la gestione del \textit{JWT Token}.

\begin{table}[H]
	\caption{Tabella dello stato di soddisfacimento degli obiettivi}
	\label{tab:obiettivi-raggiunti}
	\renewcommand{\arraystretch}{1.6}
	\begin{center}
	\begin{tabularx}{0.4\textwidth}{c|c}
		\hline\hline
		\textbf{Obiettivo} & \textbf{Descrizione}\\
		\hline
		O01 & Soddisfatto\\
		\hline
		O02 & Soddisfatto\\
		\hline
		O02 & Soddisfatto\\
		\hline
		D01 & Soddisfatto\\
		\hline
		F01 & Non soddisfatto\\
		\hline
	\end{tabularx}
	\end{center}
\end{table}%

Data l'impossibilità di soddisfare il requisito facoltativo a causa di una scelta progettuale, in accordo con l'azienda, ho prodotto le seguenti maschere non previste dal piano di lavoro:
\begin{itemize}
	\item homepage;
	\item registrazione;
	\item modifica dei dati personali;
	\item modifica della password;
	\item pagina dell'utente.
\end{itemize}

Inoltre ho iniziato la fase di \textit{testing} anch'essa non compresa nel piano di lavoro.

%**************************************************************
\section{Conoscenze acquisite}
\label{sec:conoscenze-acquisite}

Durante questo periodo di stage ritengo di aver imparato molto. Sono infatti riuscita a consolidare e migliorare le mie conoscenze riguardanti il linguaggio Javascript, inoltre ho imparato ad utilizzare il \gls{frameworkg} Vue.js in maniera soddisfacente entrando nel dettaglio imparando ad utilizzare costrutti complessi rispetto alle conoscenze basilari. Inoltre ho imparato ad utilizzare le librerie Vuetify e Vuex, la prima molto utile per creare \gls{web applicationg} con \textit{material design} e \textit{responsive} mentre la seconda è utile per effettuare le chiamate al \gls{back endg} ed utilizzare lo state management.\\
Ho potuto inoltre migliorare la mia capacità di affrontare e risolvere un problema. Se prima dello stage ero più propensa a fare affidamento totale a chi è più esperto di me, ho imparato a non fermarmi alla prima difficoltà ma a cercare per prima cosa la soluzione al problema in autonomia e, solo dopo diversi tentativi, provare a chiedere ad una persona più esperta di me in materia.\\
Infine, avendo dovuto lavorare con altri colleghi stagisti, ho migliorato ulteriormente la mia capacità di lavorare in gruppo e di comunicazione. Infatti ho capito che queste due caratteristiche sono elementi molto importanti per poter riuscire a sviluppare al meglio un prodotto software.

%**************************************************************
\section{Valutazione personale}
\label{sec:valutazione-personale}

Ritengo che questa esperienza di stage sia stata stimolante e costruttiva che mi ha permesso di crescere sia a livello personale che a livello intellettuale, affacciandomi per la prima volta al mondo del lavoro.\\
Ho avuto la possibilità di conoscere meglio me stessa, le mie capacità e i miei limiti in campo lavorativo. In questo modo ho potuto migliorare le mie conoscenze informatiche, le mie capacità pratiche e il mio approccio con i colleghi.\\
Ho apprezzato molto l'ambiente lavorativo in cui ho lavorato: sia i tutor che gli altri dipendenti dell'azienda hanno creato un ambiente tranquillo dove poter lavorare. Posso ritenermi quindi soddisfatta della scelta compiuta qualche mese fa su dove iniziare questa esperienza. 
