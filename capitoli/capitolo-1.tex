% !TEX encoding = UTF-8
% !TEX TS-program = pdflatex
% !TEX root = ../tesi.tex

%**************************************************************
\chapter{Introduzione}
\label{cap:introduzione}
%**************************************************************

Lo scopo di questo progetto di stage è studiare ed implementare tramite il framework Vue.js parte delle componenti di front-end di un'applicazione web in ambito Blockchain e NFT chiamata NFTLab. \\
Tramite questa applicazione sarà possibile creare un proprio profilo per poter compiere diverse azioni come visualizzare o vendere le proprie opere multimediali e comprare le stesse da altri utenti. \\
Per realizzare questi aspetti sono state implementate diverse maschere, ovvero interfacce utente.\\
Un aspetto importante di questo progetto è la collaborazione con gli altri componenti del gruppo per integrare gli sviluppi con quelli dei colleghi.

%Introduzione al contesto applicativo.\\

%\noindent Esempio di utilizzo di un termine nel glossario \\
%\gls{api}. \\

%\noindent Esempio di citazione in linea \\
%\cite{site:agile-manifesto}. \\

%\noindent Esempio di citazione nel pie' di pagina \\
%citazione\footcite{womak:lean-thinking} \\

%**************************************************************

\section{Organizzazione del testo}

\begin{description}
    \item[{\hyperref[cap:descrizione-stage]{Il secondo capitolo}}] approfondisce la descrizione dello stage, l'organizzazione del lavoro e il progetto da sviluppare;
    
    \item[{\hyperref[cap:nozioni-apprese]{Il terzo capitolo}}] approfondisce le nozioni apprese nel periodo di stage;
    
    \item[{\hyperref[cap:analisi-requisiti]{Il quarto capitolo}}] approfondisce il processo di analisi dei requisiti;
    
    \item[{\hyperref[cap:progettazione-codifica]{Il quinto capitolo}}] approfondisce i processi di progettazione e codifica;
    
    \item[{\hyperref[cap:verifica-validazione]{Il sesto capitolo}}] approfondisce i processi di validazione e verifica del codice prodotto;
    
    \item[{\hyperref[cap:conclusioni]{Il settimo capitolo}}] descrive le conclusioni tratte alla fine del periodo di stage.
\end{description}

Riguardo la stesura del testo, relativamente al documento sono state adottate le seguenti convenzioni tipografiche:
\begin{itemize}
	\item gli acronimi, le abbreviazioni e i termini ambigui o di uso non comune menzionati vengono definiti nel glossario, situato alla fine del presente documento;
	\item per la prima occorrenza dei termini riportati nel glossario viene utilizzata la seguente nomenclatura: \emph{parola}\glsfirstoccur;
	\item i termini in lingua straniera o facenti parti del gergo tecnico sono evidenziati con il carattere \emph{corsivo}.
\end{itemize}