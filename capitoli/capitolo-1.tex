% !TEX encoding = UTF-8
% !TEX TS-program = pdflatex
% !TEX root = ../tesi.tex

%**************************************************************
\chapter{Introduzione}
\label{cap:introduzione}
%**************************************************************

Lo scopo di questo progetto di stage è studiare il \gls{frameworkg}$_G$ Vue.js per implementare parte delle componenti di \gls{front endg}$_G$ di un'applicazione web in ambito Blockchain e \gls{NFTg}$_G$ chiamata NFTLab. \\
Tramite questa applicazione sarà possibile per l'utente creare un proprio profilo per poter compiere diverse azioni, come vendere le proprie opere multimediali, modificare alcuni dati di tali opere, visualizzare opere di altri utenti e comprare opere multimediali di altri utenti. \\
Per realizzare questi aspetti sono state implementate diverse maschere, ovvero interfacce utente, in collaborazione con gli altri colleghi stagisti.\\
Infatti un aspetto importante di questo progetto è la collaborazione con gli altri componenti del gruppo per integrare il proprio lavoro con quello degli altri, in particolare per il corretto funzionamento della \gls{web applicationg} è necessaria l'integrazione tra \gls{front endg} e \gls{back endg}$_G$.

%**************************************************************

\section{Organizzazione del testo}
\label{sec:organizzazione-testo}

\begin{description}
    \item[{\hyperref[cap:descrizione-stage]{Il secondo capitolo}}] approfondisce la descrizione dello stage, l'organizzazione del lavoro e il progetto da sviluppare;
    
    \item[{\hyperref[cap:nozioni-apprese]{Il terzo capitolo}}] approfondisce le nozioni apprese nel periodo di stage;
    
    \item[{\hyperref[cap:analisi-requisiti]{Il quarto capitolo}}] approfondisce il processo di analisi dei requisiti e il tracciamento di essi;
    
    \item[{\hyperref[cap:progettazione-codifica]{Il quinto capitolo}}] approfondisce i processi di progettazione e codifica descrivendo anche le tecnologie e gli strumenti utilizzati;
    
    \item[{\hyperref[cap:verifica-validazione]{Il sesto capitolo}}] approfondisce i processi di validazione e verifica del codice prodotto;
    
    \item[{\hyperref[cap:conclusioni]{Il settimo capitolo}}] descrive le conclusioni tratte alla fine del periodo di stage.
\end{description}

\section{Regole tipografiche}
\label{sec:regole-tipografiche}

Riguardo la stesura del testo, relativamente al documento sono state adottate le seguenti convenzioni tipografiche:
\begin{itemize}
	\item gli acronimi, le abbreviazioni e i termini ambigui o di uso non comune menzionati vengono definiti nel glossario, situato alla fine del presente documento;
	\item per la prima occorrenza dei termini riportati nel glossario viene utilizzata la seguente nomenclatura: \textcolor{blue}{parola}$_G$;
	\item per le successive occorrenze dei termini riportati nel glossario viene utilizzata la seguente nomenclatura: \textcolor{blue}{parola};
	\item i termini in lingua straniera o facenti parti del gergo tecnico sono evidenziati con il carattere \emph{corsivo}.
	\item i nomi dei file o delle cartelle sono evidenziati con il carattere corsivo grassetto \textbf{\textit{nome}}.
\end{itemize}