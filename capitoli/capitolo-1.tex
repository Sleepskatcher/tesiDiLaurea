% !TEX encoding = UTF-8
% !TEX TS-program = pdflatex
% !TEX root = ../tesi.tex

%**************************************************************
\chapter{Introduzione}
\label{cap:introduzione}
%**************************************************************

%Introduzione al contesto applicativo.\\

%\noindent Esempio di utilizzo di un termine nel glossario \\
%\gls{api}. \\

%\noindent Esempio di citazione in linea \\
%\cite{site:agile-manifesto}. \\

%\noindent Esempio di citazione nel pie' di pagina \\
%citazione\footcite{womak:lean-thinking} \\

%**************************************************************
\section{L'azienda}

Sync Lab S.r.l. è un'azienda di consulenza informatica nata nel 2002 nella sede di Napoli. L'azienda si è evoluta molto velocemente nel corso degli anni trasformandosi in System Integrator grazie ad un processo di maturazione delle competenze tecnologiche. Inoltre l'azienda è riuscita a coprire tutto il territorio nazionale fino ad ottenere un totale di cinque sedi a Roma, Napoli, Verona, Padova e Milano.\\
Ad oggi l'azienda conta un organico di circa 200 dipendenti e numerosi clienti importanti come Unicredit, Poste Italiane, Fastweb e Sky. \\
L'azienda, propone sul mercato prodotti software, attraverso essi Sync Lab ha gradualmente conquistato significativamente fette di mercato nei seguenti settori: mobile, videosorveglianza e sicurezza delle infrastrutture informatiche aziendali.
Il loro obiettivo é quello di supportare il cliente nella Realizzazione, Messa in Opera e Governance di soluzioni IT, sia dal punto di vista Tecnologico, sia nel Governo del Cambiamento Organizzativo.
\begin{figure}[h]
	\begin{center}
		\includegraphics[width=7cm]{logo_large}
	\end{center}
\end{figure}

%**************************************************************
\section{L'idea dello stage}

Introduzione all'idea dello stage.

%**************************************************************
\section{Organizzazione del testo}

\begin{description}
    \item[{\hyperref[cap:processi-metodologie]{Il secondo capitolo}}] descrive i processi e le metodologie di sviluppo;
    
    \item[{\hyperref[cap:descrizione-stage]{Il terzo capitolo}}] approfondisce la descrizione dello stage e del progetto da sviluppare;
    
    \item[{\hyperref[cap:analisi-requisiti]{Il quarto capitolo}}] approfondisce il processo di analisi dei requisiti;
    
    \item[{\hyperref[cap:progettazione-codifica]{Il quinto capitolo}}] approfondisce i processi di progettazione e codifica;
    
    \item[{\hyperref[cap:verifica-validazione]{Il sesto capitolo}}] approfondisce i processi di validazione e verifica del codice prodotto;
    
    \item[{\hyperref[cap:conclusioni]{Il settimo capitolo}}] descrive le conclusioni tratte alla fine del periodo di stage.
\end{description}

Riguardo la stesura del testo, relativamente al documento sono state adottate le seguenti convenzioni tipografiche:
\begin{itemize}
	\item gli acronimi, le abbreviazioni e i termini ambigui o di uso non comune menzionati vengono definiti nel glossario, situato alla fine del presente documento;
	\item per la prima occorrenza dei termini riportati nel glossario viene utilizzata la seguente nomenclatura: \emph{parola}\glsfirstoccur;
	\item i termini in lingua straniera o facenti parti del gergo tecnico sono evidenziati con il carattere \emph{corsivo}.
\end{itemize}