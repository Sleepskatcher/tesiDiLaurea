% !TEX encoding = UTF-8
% !TEX TS-program = pdflatex
% !TEX root = ../tesi.tex

%**************************************************************
\chapter{Appendice A}
%**************************************************************
%\epigraph{Citazione}{Autore della citazione}

\section{Confronto tra framework di sviluppo}

Durante il mio progetto di stage un altro stagista ha sviluppato le stesse maschere front end come quelle sviluppate da me, ma tramite un \gls{frameworkg} differente: \gls{Angular.jsg}$_G$. Grazie a questo, nell'ultima settimana in cui stavamo ultimando le diverse implementazioni e l'integrazione, abbiamo potuto fare un confronto fra i due \gls{frameworkg}.\\
Le differenze che abbiamo trovato possono essere riassunte nella seguente tabella.

\begin{table}[H]
	\caption{Tabella riassuntiva delle differenze tra Vue.js ed Angular.js}
	\label{tab:confronto-framework}
	\renewcommand{\arraystretch}{1.6}
	\begin{tabularx}{\textwidth}{lX|X}
		\hline\hline
		\textbf{} & \textbf{Vue.js} & \textbf{Angular.js}\\
		\hline
		Curva di apprendimento & Bassa & Alta \\
		\hline
		Linguaggio & Javascript & Typescript \\
		\hline
		Comunity & Poco ampia ma in crescita & Molto ampia ma sta morendo un po' alla volta \\
		\hline
		Two-way data binding & Assente & Presente \\
		\hline
		Organizzazione dei file & Per ogni componente esiste un solo file contente il codice della parte logica, html e stile & Per ogni componente si crea una cartella contenente un file per parte logica, html e stile \\
		\hline
		Peso alla creazione del progetto & Nel momento in cui si esegue per creare il progetto viene creata una cartella di peso pari molto esiguo & Nel momento in cui si esegue per creare il progetto viene creata una cartella di peso consistente \\
		\hline
	\end{tabularx}
\end{table}%

Inoltre il \gls{frameworkg} Vue.js viene utilizzato maggiormente per lo sviluppo di web application destinate ad un bacino di utente ampio mentre \gls{Angular.jsg} viene utilizzato per lo sviluppo enterprise, ovvero per lo sviluppo di web application da utilizzare all'interno di aziende.\\
Abbiamo notato come, a lato pratico, Vue.js fosse molto più veloce e semplice da utilizzare rispetto ad \gls{Angular.jsg}, ad esempio per creare una finestra di dialogo si riscontra meno difficoltà su Vue.js rispetto che su \gls{Angular.jsg}.\\
Inoltre Vue.js possiede il plugin Vuetify che aiuta a creare delle \gls{web applicationg} responsive ed utilizzabili anche su risoluzioni inferiori a quelle dello schermo da computer senza scrivere codice di stile. Questa cosa abbiamo notato essere assente in \gls{Angular.jsg}.



