% !TEX encoding = UTF-8
% !TEX TS-program = pdflatex
% !TEX root = ../tesi.tex

%**************************************************************
\chapter{Nozioni apprese}
\label{cap:nozioni-apprese}
%**************************************************************

\intro{In questo capitolo verranno elencate e spiegate brevemente le nozioni apprese durante questo percorso di stage.}

%**************************************************************
\section{Organizzazione dello studio}

Le ore complessive dello stage possono essere considerate suddivise in 3 periodi:
\begin{itemize}
	\item periodo di studio della durata di circa 160 ore;
	\item periodo di sviluppo della durata di circa 120 ore;
	\item peridio di convalidazione finale della durata di circa 40 ore.
\end{itemize}

Nelle sezioni qui di seguito verranno elencate e spiegate le nozioni apprese nel periodo di studio.\\
Per chiarezza sono state suddivise in principali e secondarie, non per dare meno importanza ad alcuni argomenti studiati ma per suddividerli in funzione dello sviluppo del progetto. Infatti le prime mi sono maggiormente servite per poter sviluppare le maschere, mentre le seconde mi sono servite per conoscenza generale e migliore collaborazione con i colleghi addetti al reparto back end.

%**************************************************************
\section{Nozioni principali}

Le nozioni principali da apprendere legate a questo progetto furono:
\begin{itemize}
	\item Javascript;
	\item Vue.js;
\end{itemize}


%**************************************************************
\section{Nozioni secondarie}

Le nozioni secondarie da apprendere legate a questo progetto furono:
\begin{itemize}
	\item Java;
	\item Concetti web;
	\item Spring;
	\item Blockchain;
	\item Ethereum;
	\item NFT.
\end{itemize}

