% !TEX encoding = UTF-8
% !TEX TS-program = pdflatex
% !TEX root = ../tesi.tex

%**************************************************************
\chapter{Nozioni apprese}
\label{cap:nozioni-apprese}
%**************************************************************

\intro{In questo capitolo verranno elencate e spiegate brevemente le nozioni apprese durante questo percorso di stage.}

%**************************************************************
\section{Organizzazione dello studio}

Le ore complessive dello stage possono essere considerate suddivise in 3 periodi:
\begin{itemize}
	\item perdio di studio;
	\item periodo di sviluppo;
	\item peridio di convalidazione finale.
\end{itemize}

%**************************************************************
\section{Nozioni principali}

Durante la fase di analisi iniziale sono stati individuati alcuni possibili rischi a cui si potrà andare incontro.
Si è quindi proceduto a elaborare delle possibili soluzioni per far fronte a tali rischi.\\

\begin{risk}{Performance del simulatore hardware}
    \riskdescription{le performance del simulatore hardware e la comunicazione con questo potrebbero risultare lenti o non abbastanza buoni da causare il fallimento dei test}
    \risksolution{coinvolgimento del responsabile a capo del progetto relativo il simulatore hardware}
    \label{risk:hardware-simulator} 
\end{risk}

%**************************************************************
\section{Nozioni secondarie}
