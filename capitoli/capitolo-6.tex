% !TEX encoding = UTF-8
% !TEX TS-program = pdflatex
% !TEX root = ../tesi.tex

%**************************************************************
\chapter{Verifica e validazione}
\label{cap:verifica-validazione}
%**************************************************************

\textit{In questo capitolo verranno descritti gli strumenti e i processi utilizzati per la verifica del codice.}

\section{Strumenti per la verifica}
\label{sec:strumenti-per-verifica}

Per verificare il codice prodotto ho utilizzato la libreria ufficiale di test Vue Test Utils. Grazie a questa libreria si possono testare le interazioni dell'utente, come un click di un bottone, le chiamate al \gls{back endg}, le routes e l'utilizzo di Vuex.

\section{Verifica}
\label{sec:verifica}

Per ogni componente ho creato un file che, nella fase di run, viene chiamato \textit{test suit}. Ogni \textit{text suit} può contenere al suo interno diversi test.
\begin{lstlisting}[caption=Esempio di test-Pagina di login., label=lst::esTest]
	describe('login.vue', () => {
		let vuetify;
		beforeEach(() => {
			vuetify = new Vuetify();
		});
		const wrapper = shallowMount(login, {
			localVue,
			store,
			vuetify
		});
		it('Check if content render', () => {
			expect(wrapper.contains('v-container')).toBe(true);
		});
		it('Check if button is disabled with empty fields', ()=> {
			const email="";
			const password="";
			
			var emailInput = wrapper.find('#emailInput');
			emailInput.element.value = email;
			var passwordInput = wrapper.find('#passwordInput');
			passwordInput.element.value = password;
			
			expect(wrapper.vm.isFormValid).toBeFalsy();
			expect(wrapper.find('v-btn').element.hasAttribute('disabled')).not.toBe(true);
		});
		it('Check if email is changed after user interaction', () =>{
			var emailInput = wrapper.find('#emailInput');
			wrapper.find('#emailInput').trigger('click');
			wrapper.find('#emailInput').trigger('input');
			expect(emailInput.value).not.toBe('');
		});
		it('Check if old password is changed after user interaction', () =>{
			var passwordInput = wrapper.find('#passwordInput');
			wrapper.find('#passwordInput').trigger('click');
			wrapper.find('#passwordInput').trigger('input');
			expect(passwordInput.value).not.toBe('');
		});
	});
\end{lstlisting}
Prendiamo come esempio lo snippet di codice soprastante che è il \textit{test suit} per la pagina di login.
Per prima cosa viene effettuato il mock del componente tramite il comando \textit{shallowMount} dove vengono anche specificati i vari plugin e lo store utilizzati. Ogni \textit{it} presente nel codice rappresenta un singolo test.\\
Il primo test che ho effettuato è il controllo che la pagina renderizzi correttamente i suoi contenuti.
I controlli successivi che ho effettuato sono invece legati all'interazione dell'utente con gli elementi presenti nella pagina. I controlli che eseguo sono ad esempio che il bottone sia disabilitato se l'utente non compila correttamente tutti i campi della form, oppure che i campi della form vengano compilati dopo l'interazione dell'utente.\\
Per ogni componente ho effettuato test analoghi a quello appena descritto.

\section{Validazione e collaudo}

Infine durante le ultime settimane ho effettuato in collaborazione con un collega stagista l'integrazione al \gls{back endg} risolvendo possibili bug dovuti alle incongruenze tra il back end effettivo e quello simulato tramite stoplight.\\
Inoltre ho prodotto il documento tecnico richiesto dall'azienda riguardante il mio lavoro in modo che chiunque arrivi a prendere in mano il mio codice possa capire la logica che regge il mio lavoro.\\
Infine io e gli altri stagisti che hanno svolto lo stage nello stesso periodo abbiamo preparato una breve presentazione con demo effettiva del lavoro svolto da mostrare l'ultimo giorno a tutti i dipendenti disponibili dell'azienda.
