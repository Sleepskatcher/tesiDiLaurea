
%**************************************************************
% Acronimi
%**************************************************************
\renewcommand{\acronymname}{Acronimi e abbreviazioni}

\newacronym[description={\glslink{apig}{Application Program Interface}}]
    {api}{API}{Application Program Interface}

\newacronym[description={\glslink{umlg}{Unified Modeling Language}}]
    {uml}{UML}{Unified Modeling Language}
    
\newacronym[description={\glslink{NFTg}{Non-Fungible Token}}]
	{NFT}{NFT}{Non-Fungible Token}
	
\newacronym[description={\glslink{CRUDg}{Create Read Update Delete}}]
	{CRUD}{CRUD}{Create Read Update Delete}

%**************************************************************
% Glossario
%**************************************************************
\renewcommand{\glossaryname}{Glossario}

\newglossaryentry{apig}
{
    name=\glslink{api}{API},
    text=Application Program Interface,
    sort=api,
    description={in informatica con il termine \emph{Application Programming Interface API} (ing. interfaccia di programmazione di un'applicazione) si indica ogni insieme di procedure disponibili al programmatore, di solito raggruppate a formare un set di strumenti specifici per l'espletamento di un determinato compito all'interno di un certo programma. La finalità è ottenere un'astrazione, di solito tra l'hardware e il programmatore o tra software a basso e quello ad alto livello semplificando così il lavoro di programmazione}
}

\newglossaryentry{umlg}
{
    name=\glslink{uml}{UML},
    text=UML,
    sort=uml,
    description={in ingegneria del software \emph{UML, Unified Modeling Language} (ing. linguaggio di modellazione unificato) è un linguaggio di modellazione e specifica basato sul paradigma object-oriented. L'\emph{UML} svolge un'importantissima funzione di ``lingua franca'' nella comunità della progettazione e programmazione a oggetti. Gran parte della letteratura di settore usa tale linguaggio per descrivere soluzioni analitiche e progettuali in modo sintetico e comprensibile a un vasto pubblico}
}

\newglossaryentry{stakeholderg}
{
	name={stakeholder},
	description={In ingegneria del software uno stakeholder è una persona a vario titolo coinvolta nel ciclo di vita di un software, che ha influenza sul prodotto o sul processo}
}

\newglossaryentry{scrumg}
{
	name={Scrum},
	description={è un framework agile utilizzato per la gestione del ciclo di vita del software, iterativo ed incrementale, con lo scopo di gestire progetti, prodotti software o applicazioni di sviluppo}
}

\newglossaryentry{NFTg}
{
	name=\glslink{NFT}{NFT},
	text=NFT,
	sort=NFT,
	description={è un tipo speciale di token crittografico che rappresenta qualcosa di unico, i token non fungibili non sono interscambiabili. Questo concetto va in contrasto con le criptovalute che sono di natura fungibili.}
}

\newglossaryentry{CRUDg}
{
	name=\glslink{CRUD}{CRUD},
	text=CRUD,
	sort=CRUD,
	description={in infromatica CRUD, acronimo di create read update delete, sono le quattro operazini di un databaase persistente. Nel contesto dello sviluppo di applicazioni web corrispondono alle quattro operazioni HTTP: put, get, post e delete.}
}

\newglossaryentry{frameworkg}
{
	name={framework},
	description={architettura logica di supporto sulla quale un software può essere progettato e sviluppato, facilitandone lo sviluppo da parte del programmatore}
}

\newglossaryentry{front endg}
{
	name={front end},
	description={parte del programma visibile all'utente con cui egli può interagire, tipicamente è un'interfaccia utente}
}

\newglossaryentry{back endg}
{
	name={back end},
	description={parte del programma non visibile all'utente che permette il funzionamento delle interazioni con il front end}
}

\newglossaryentry{Angular.jsg}
{
	name={Angular.js},
	description={framework sviluppato da Google per lo sviluppo delle applicazioni web}
}